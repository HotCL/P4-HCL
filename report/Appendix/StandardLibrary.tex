\section{Standard Library Functions}
\label{standardLib}
HCL includes multiple functions as part of its standard library.
All standard library functions are implemented exclusively using the HCL syntax and builtin-functions. 

%\textbf{Renamed Functions}\\
%Functions stated in table \ref{tbl:renamed} are merely renamings of built-in functions for alternative naming. 
%
%\begin{table}[h!]
%	\centering
%	\caption{Renamed functions}
%	\label{tbl:renamed}
%	\begin{tabular}{|c|c|}
%		\hline
%		HCL Type & C++ Equivalent \\ \hline
%		num      & double         \\ \hline
%		txt      & string         \\ \hline
%		bool     & bool           \\ \hline
%		tuple    & struct         \\ \hline
%		list     & list           \\ \hline
%		func     & lambda         \\ \hline
%		none     & void           \\ \hline
%	\end{tabular}
%\end{table}

Snippet \ref{lis:st-to} shows the 'to' function used to create a range of numbers. Returns list of elements within range.
\begin{lstlisting}[language=Kotlin,label=lis:st-to,caption=\texttt{to} function.]
var to (num from, num to):list[num] {
	num i = from
	list[num] output
	{
		output = output + [i]
		i = i + 1
	} while { i < to }
}
\end{lstlisting}

Snippet \ref{lis:st-any}, returns true if any element in a list uphold compare predicate. Returns false if no element in list upholds compare predicate.
\begin{lstlisting}[language=Kotlin,label=lis:st-any,caption=\texttt{any} function.]
var any (list[T] myList, func[T,bool] compareFunc):bool{
	myList filter :compareFunc length greaterThan 0
}
\end{lstlisting}

Snippet \ref{lis:st-all} returns true if all elements in a list uphold compare predicate. Returns false if any element does not uphold compare predicate.
\begin{lstlisting}[language=Kotlin,label=lis:st-all,caption=\texttt{all} function.]
var all (list[T] myList, func[T,bool] compareFunc):bool{
	myList filter :compareFunc length equals myList length
}
\end{lstlisting}

Snippet \ref{lis:st-in} returns true if an element is present in a list. Returns false if not present. 
\begin{lstlisting}[language=Kotlin,label=lis:st-in,caption=\texttt{in} function.]
var in = (T element, list[T] myList):bool{
	myList any { value is element }
}
\end{lstlisting}

Snippet \ref{lis:st-notin} returns true if an element is not present within a list. Returns false if element is present in list. 
\begin{lstlisting}[language=Kotlin,label=lis:st-notin,caption=\texttt{notIn} function.]
var notIn = (T element, list[T] myList):bool{
	element in myList not
}
\end{lstlisting}

Snippet \ref{lis:st-then} is used to create a ternary statement the \texttt{else} function. Returns a tuple with types \textit{bool} and \textit{T}
\begin{lstlisting}[language=Kotlin,label=lis:st-then,caption=\texttt{then} function.]
var then = (bool condition, func[T] body): tuple[bool,T] {
	return (condition,body)
}
\end{lstlisting}

Snippet \ref{lis:st-else} shows \texttt{else} function used together with \texttt{then} function. Returns function-body dependent on result from \texttt{then} function.
\begin{lstlisting}[language=Kotlin,label=lis:st-else,caption=\texttt{else} function.]
var else = (tuple[bool,T] thenResult, func[T] body): T {
	T output
	thenResult at 0 then { output = thenResult at 1 }
	thenResult at 0 not then { output = body }
	return output
}
\end{lstlisting}

Snippet \ref{lis:st-thenElse} shows \texttt{thenElse} function that uses the functionality of both \texttt{then} and \texttt{else} functions to simulate classical "if"-statement. Returns function-body dependent on condition-evaluation. 
\begin{lstlisting}[language=Kotlin,label=lis:st-thenElse,caption=\texttt{thenElse} function.]
var thenElse = (bool condition, func[T] trueBody, func[T] falseBody): T {
	condition then {output = trueBody} else { output = falseBody}
}
\end{lstlisting}