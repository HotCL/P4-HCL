% This document needs to include:
% - An introduction for the concern.
% - Bases for section, specifically EBNF in previous section.
% - Difference between EBNF CFG.
% - HCL as CFL and HCL's CFG.
% - Proof that HCL is a CFL.
% - Proof that shows whether HCL is a regular language or not.
% - Description of identifiers and types in HCL using computational models or Regex.
% - Conclusion.
\section{Mathematical Syntax Theory}

\subsection{Types and Identifiers}
So far the syntax of HCL has been properly defined as a CFL. 
Meaning that all Variables, terminals and the transition rules for recognizing HCl has been described in a computational model as seen on page 
%Indsæt sidetal for HCL CFG definition.
and in appendix.
%Indæst reference til appendix hvor NPDA for HCL er beskrevet.
It is now relevant to define how the types and identifiers in HCL can be described or recognized by a computational model.
The goal is to formally describe the languages $L_{none}$, $L_{bool}$, $L_{number}$, $L_{text}$ which are the type languages and the identifier language $L_{identifiers}$.

An alphabet is a finite set of symbols, a string is a finite sequence of symbols and a language is a set of strings.
To describe the languages mentioned above, the Alphabets used in the informal and formal notations needs to be defined. These can be seen in Table 3.1.

\begin{table}[!htb]
	\centering
	\begin{tabular}{|l|l|}
		\hline
		\textbf{Alphabet} & \textbf{Set}                            \\ \hline
		T                 & \{true\}                                \\ \hline
		F                 & \{false\}                               \\ \hline
		$Z^+$             & The set of all positive integers        \\ \hline
		$Z^-$             & The set of all negative integers        \\ \hline
		R                 & The set of all real numbers             \\ \hline
		D                 & \{$Z^+ \cup Z^-$\}                      \\ \hline
		$S_p$             & The set of all special characters.      \\ \hline
		$L_l$             & The set of all lower-case letters       \\ \hline
		$L_u$             & The set of all upper-case letters       \\ \hline
		C                 & $S_p \cup L_l \cup L_u \cup D$          \\ \hline
		$R_{eserved}$       & The set of all reserved keywords in HCL \\ \hline
	\end{tabular}
	\caption{Table containing all utility alphabets used to formally describe the type and identifier languages.}
\end{table}

By using the set construction method, the languages can now be defined as shown in Table 3.2.

\begin{table}[!htb]
	\centering
	\label{my-label}
	\begin{tabular}{|l|l|}
		\hline
		\textbf{Name}     & \textbf{Set}                                    \\ \hline
		$L_{none}$        & \{w | w $\in$ \{none\}\}                        \\ \hline
		$L_{bool}$        & \{w | w $\in$ (T $\cup$ F)\}                    \\ \hline
		$L_{number}$      & \{w | w $\in$ R\}                               \\ \hline
		$L_{text}$        & \{"w" | all strings from arbitrary alphabets\}  \\ \hline
		$L_{identifiers}$ & \{w | w $\in (L_{text} - \{", "\}) \wedge w \notin R_{eserved}$\} \\ \hline
	\end{tabular}
	\caption{Table containing the type and identifier alphabets described by the set builder construction}
\end{table}

Intuitively it is trivial that all of the languages defined above are regular languages.
This is the case since a regular language cannot keep track of more than one symbol in terms of counting and iterations.
Since all of the languages in Table 3.2 consists only of arbitrary concatenations of elements of various alphabets, this means that they are indeed regular. 
Now that it has been determined that the languages are regular, a computational model can now be chosen to describe them more specifically.
A regular expression for each of the languages will now be constructed.

For the sake of convenience $R^+$ will be used at a shorthand for $RR^*$ \footnote{Where * is Kleene star}, meaning that $R^+ \cup \epsilon = R^*$ since $R^*$ consists of all strings that are zero or more concatenations from R and $RR^*$ consists of all strings that are one or more concatenations of strings from R.
The resulting regular expressions are shown in Table 3.3.

\begin{table}[!htb]
	\centering
	\label{my-label}
	\begin{tabular}{|l|l|}
		\hline
		\textbf{Name}     & \textbf{Regular Expression}         \\ \hline
		$L_{none}$        & none                                \\ \hline
		$L_{bool}$        & true $\cup$ false                   \\ \hline
		$L_{number}$      & $D^+(.D^+)^*$                       \\ \hline
		$L_{text}$        & $"C^*"$                               \\ \hline
		$L_{identifiers}$ & $C^+(R_{eserved} \cup \epsilon)C^+$ \\ \hline
	\end{tabular}
	\caption{Regular Expressions describing the type and identifier languages.}
\end{table}



