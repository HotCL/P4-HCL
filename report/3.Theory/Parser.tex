\section{Parser}

The second part of the compiler will parse the lexical tokens coming from the lexer. 
This part of the compiler is called the parser. 
The parser works by taking a sequence of lexical tokens, and generating a parse tree or transforming them into an abstract syntax tree. 

\subsection{Output data types}

The parser will output some data type, that represents the compiled program in a hierarchical structure. 
The outputted data type should hold all the information or almost all the information needed to generate the code later. 
Sometimes the data type will be accommodated by a symbol table in order to generate the desired code.

\textbf{Parse tree}

One of the possible data structures to be outputted from the parser is the parse tree explained in section \ref{CFGdescription}. 
The parse tree should resemble the structure of the CFG and the concrete input to the parser. 
The parse tree is easily generated, and often the actual output from the parser. 
However, the parse tree contains a lot of excessive information, which is not needed to understand the complete meaning of the program. 
This means that a parse tree is both hard to analyze and slow to traverse. 
Therefore, another common output type from the parser is the abstract syntax tree.

\textbf{Abstract syntax tree}

The abstract syntax tree is a sparse version of the parse tree. 
It still contains all the information needed to understand the structure of the program, but in a more compressed way. 
For instance parentheses can be omitted in the abstract syntax tree.

Recall the parse tree from figure \ref{fig:AmbiguousGrammarEx1} a corresponding abstract syntax tree can bee seen on figure \ref{fig:ast} 

\begin{figure}[H]
	\centering
	\includegraphics[width=\textwidth/2+\textwidth/4]{3.Theory/images/ast.png}
	\caption{
		Abstract syntax tree for parse tree in figure \ref{fig:AmbiguousGrammarEx1}.
	}
	\label{fig:ast}
\end{figure}

%\subsection{Grammar}
%
%The parser is built upon the CFG defined for the language. 
%It is the responsibility of the CFG to tell the parser how to parse the language. 
%For instance, when writing a top down parser, first and follow sets can be generated on behalf of the CFG. 
%These will help determine what should be parsed next. 
%%%%%%%%%%%%%%%%%%%%%%%%%%%%
% I guess it is not really necessary to talk about CFG, when we already have a big section about it.
%%%%%%%%%%%%%%%%%%%%%%%%%%%%

\subsection{Contextual analysis}

Apart from syntactical analysis the parser is also concerned with contextual analysis. 
The contextual analysis ensures that types are correct, immutable variables are not reassigned and other contextual things that cannot be inferred from the CFG due to its context free nature. 
The major part of the contextual analysis is the type checking.

\textbf{Type checking}

The type checking part of the parser often utilizes a symbol table.

INSERT JONAS SYMBOL TABLE STUFF HERE.

It is also the responsibility of this part of the parser to provide any relevant type information that may be needed in order to proceed with compilation in later phases. 
There are several ways to inform the later stages of the compiler about the needed type information. 
One solution is to store and pass the symbol table to the later stages of the compiler. 

Another way is to decorate the abstract syntax tree with the needed type information. 
This is called a decorated syntax tree, and will in most cases be the final output of the parser.
An example of decorating an abstract syntax tree could be when using implicit casts. 
Most languages have this feature between integers and floating point numbers. 
So if the user were to do arithmetic with an integer and a floating point number, the integer would then be implicitly casted to a floating point number.

\subsection{Single parse and multi parse}

Depending on the language, a single run through the passer may not be enough to validate the input or generate the output. 
For instance if the language allows calling of functions or usage of classes that has not yet been declared. 
In this case the parser should do a single parse to map out the functions and classes that can be used. 
After that another parse should be done, where the information from the previous parse is utilized to ensure the validity of the input. 

Surely single pass parsers are the fastest. 
However it does limit the language a bit, and therefore most mainstream compilers are multi parse compilers these days.

\subsection{Parser types}

Parsers comes in a wide variety of different flavors. 
Generally there are two different approaches to creating a parser.

\textbf{Top down parsing}

LL parsers uses the top down parsing strategy. 
LL means that the parser is Left to right using Left most derivation. 
If a CFG is LL(K) it means that the parser needs to do at max K look aheads in order to parse the sentence. 
LL parsers can be implemented as a table parser, or also often as a recursive decent parser. 

\textbf{Bottom up parsing}

There are a couple of different bottom up parsers. 
Namely the LR(K) and the LALR parsers are interesting. 
The bottom up parsers can recognize a lot more languages than the top down parsers.
However, they are also generally more complex to implement.
Because of this bottom up parsers are only rarely handmade.
They will usually be implemented using the shift reduce strategy.


\subsection{Parser tools}

Like for the lexer, a vast amount of tools exist in order to ease the parsing part of the compiler. 
The complexity of the parser depends a lot on the type of language. 
As mentioned a LL(1) recursive decent parser i fairly simple to write whereas a LALR parser is quite tedious to write. 
Since only a few languages are simple enough to have a handwritten parser pay off, most compilers will often use tool generated parsers.

One of the most popular choices for parser generators are the previously mentioned ANTLR parser generator. 
Another common choice is SableCC which is a LALR parser generator.
SableCC accepts EBNF grammar, and can produce among other things java compatible code.