%Content includes all type rules in HCL
%Should be a relatively short section, since much of the content is self explanatory.
\section{Type Rules}
To ensure that any program written in HCL runs correctly after compilation, the language follows certain type rules.
These rules specify where expressions have to evaluate to a specific type.
There are essentially two cases where type rules apply:
\begin{itemize}
	\item When assigning an expression to a variable
	\begin{itemize}
		\item the expression must evaluate to the same type as the variable.
	\end{itemize}
	\item When passing an expression as an argument a function call.
	\begin{itemize}
		\item the expression must evaluate to the type specified by the function declaration.
	\end{itemize}
\end{itemize}


\subsection{Type Rules for Expressions}
%quick introduction.
\textbf{Addition}\\
\begin{center}
	\begin{math}
	\cfrac
	{E \vdash e_1: T \quad E \vdash e_2: T}
	{E \vdash e_1 + e_2 : T}
	\end{math}\\[1\baselineskip]
	\texttt{where} $T \in \{num, txt, list\}$
\end{center}

\textbf{Subtraction}\\
\begin{center}
	\begin{math}
	\cfrac
	{E \vdash e_1: num \quad E \vdash e_2: num}
	{E \vdash e_1 - e_2 : num}
	\end{math}
\end{center}

\textbf{Multiplication}\\
\begin{center}
	\begin{math}
	\cfrac
	{E \vdash e_1: num \quad E \vdash e_2: num}
	{E \vdash e_1 * e_2 : num}
	\end{math}
\end{center}

\textbf{Division}\\
\begin{center}
	\begin{math}
	\cfrac
	{E \vdash e_1: num \quad E \vdash e_2: num}
	{E \vdash e_1 / e_2 : num}
	\end{math}
\end{center}

\textbf{Less than}\\
\begin{center}
	\begin{math}
	\cfrac
	{E \vdash e_1: num \quad E \vdash e_2: num}
	{E \vdash e_1 < e_2 : bool}
	\end{math}
\end{center}

\textbf{Greater than}\\
\begin{center}
	\begin{math}
	\cfrac
	{E \vdash e_1: num \quad E \vdash e_2: num}
	{E \vdash e_1 > e_2 : bool}
	\end{math}
\end{center}

\textbf{Equals}\\
\begin{center}
	\begin{math}
	\cfrac
	{E \vdash e_1: T \quad E \vdash e_2: T}
	{E \vdash e_1\ equals\ e_2 : bool}
	\end{math}\\[1\baselineskip]
	\texttt{where} $T \in \{num, bool, txt, list\}$
\end{center}

\textbf{And}\\
\begin{center}
	\begin{math}
	\cfrac
	{E \vdash e_1: bool \quad E \vdash e_2: bool}
	{E \vdash e_1\ and\ e_2 : bool}
	\end{math}
\end{center}

\textbf{Or}\\
\begin{center}
	\begin{math}
	\cfrac
	{E \vdash e_1: bool \quad E \vdash e_2: bool}
	{E \vdash e_1\ or\ e_2 : bool}
	\end{math}
\end{center}


\subsection{Type Rules for Statements}
%quick introduction.

%Conclusion.

