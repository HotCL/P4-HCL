\section{Parser implementation}
\label{parserImplemention}
In the following section, the parser implementation details will be touched upon.

Due to the nature of HCL, the parser can be implemented as a single pass parser. 
This is because the user may not access any variables that has not yet been declared. 
As HCL is a LL(K) language, an LL(k) recursive decent parser is a reasonable choice.

Both syntactical and contextual analysis is done within the parser. 
This means that there is no additional post parser type-checking or contextual analysis step. 
A major reason for doing in line contextual analysis is the way functions are invoked in HCL. 
The only way to know if an identifier is a function call, is by analyzing the types. 
This means that to create a proper AST, the type-checking must be done in line. 
It could also potentially be done afterwards, but that would require quite a large amount of post processing, which would be a lot more complex, than just doing type-checking within the parser. 
This also means that the outputted AST is in fact already a decorated syntax tree. 
For instance, function calls already has information on its return type, even though the function may have multiple return types based on the overloading of the function.

\subsection{Implementing a recursive decent parser}
A LL(K) recursive decent parser will do lookaheads until it is aware what branch should be parsed.
This can be figured out from the first and follow sets, that can be produced from the CFG.
Once it determines which non-terminal to parse it will call a method that attempts to parse that part.
The method can then accept terminals or parse further non-terminals.
It should be the responsibility of the accept method to advance the lexical token stream.

Below is a quick and concrete example, where the parser generates a token stream of the following HCL source code:
\begin{lstlisting}[language=HCL,label=lis:typedcls,firstnumber=1]
num x = 5
\end{lstlisting}

\begin{enumerate}
	\item Type.Number
	\item Identifier("x")
	\item SpecialChar.Equals
	\item Literal.Number(5.0)
	\item SpecialChar.EndOfLine
\end{enumerate}

The parser will see that the first token is a number type. 
Since only declarations have types in its first set, the parser now knows that it should parse a declaration.

The EBNF reveals that a declaration is:
\begin{itemize}
	\item Non-terminal Type
	\item Terminal Identifier
	\item Optional
	\begin{itemize}
		\item Terminal Equals
		\item Non-terminal Expression
	\end{itemize}
\end{itemize}
\textbf{Deciding what to parse}\\
The sealed classes, along with the pattern matching of Kotlin, are excellent tools in deciding what should be parsed. 
In snippet \ref{lis:parseCommand} a subset of the parse command method is shown. 

\begin{lstlisting}[language=Kotlin,label=lis:parseCommand,caption=A simplified version of the parse declaration method from the parser.]
private fun parseCommand(): AstNode.Command {
val command = when (current.token) {
is Token.Type -> parseDeclaration()
is Token.Identifier ->
if (peek().token == Token.SpecialChar.Equals) parseAssignment() 
else parseExpression()

...

flushNewLine()
return command
}
\end{lstlisting}
Snippet \ref{lis:parseCommand} uses the functional functionality of Kotlin, to set the val\footnote{val is a read only variable} \textit{\textbf{command}} to a specific val based on the type of the current token (\textit{\textbf{current.token}}). 
This method calls other methods based on the type and based on specific rules in the HCL language. 
For instance all statements starting with a "type" is a declaration, as seen in the CFG, therefore the \textit{\textbf{parseDeclaration}} method is called, and the returned val of this method is then going to be assigned to the command val.
	
To simplify the parser and increase readability, certain helper methods is utilized. 
For instance the \textit{\textbf{accept<T>}} method. 
\begin{lstlisting}[language=Kotlin,label={lis:acceptMethod},caption=The definition of the accept method on the parser]
private inline fun<reified T: Token> accept(): T {
	val token = current.token
	moveNext()
	if (token is T) {
		return token
	} else {
		wrongTokenTypeError(T::class.simpleName!!, token)
	}
}
\end{lstlisting}
Snippet \ref{lis:acceptMethod} shows how the accept method is defined.
This method is used to pop a token from the token stream, which is done by the \textit{\textbf{moveNext()}} method, if the token has the expected type.
This is used for getting the current token if it's of the correct type. 
This can also be used even if the data of the token is irrelevant.
For instance shown in snippet \ref{lis:acceptMethodHCLExample}, the accept method would be used to pop the equals-symbol. 
However it's important that the equals symbol is present, as the code would not make sense without it. 
The accept method would then trow and error if another token would be present.

\begin{lstlisting}[language=HCL,label={lis:acceptMethodHCLExample},caption={An example of a declaration in HCL, where the accept method is used by the parser.}]
num a = 5
\end{lstlisting}
Snippet \ref{lis:acceptMethodHCLExample} is a declaration, and these are handled by the \textit{\textbf{parseDeclaration}}, which is implemented as seen in snippet \ref{lis:parseDeclaration}. This is a great example of how parsing is done on a specific type of statement.

Do note, the final parse declaration method is a little more complex due to type inference and type checking.

\begin{lstlisting}[language=Kotlin,label=lis:parseDeclaration,caption=A simplified version of the parse declaration method from the parser.]
private fun parseDeclaration(): AstNode.Command.Declaration {
    val type = parseType()
    val identifierToken = accept<Token.Identifier>()
    val identifier = AstNode.Command.Expression.Value.Identifier(identifierToken.value)
    val expression = if (current.token == Token.SpecialChar.Equals) {
	    accept<Token.SpecialChar.Equals>()
	    parseExpression()
    } else null
    return AstNode.Command.Declaration(type, identifier, expression)
}
\end{lstlisting}

The \textbf{\textit{parseDeclaration}} method, as shown in snippet \ref{lis:parseDeclaration}, will first save the type in the \textbf{\textit{type}} val, and then accept an identifier and saving it to an instance of Identifier, a subclass of the AstNode class. 
the method then checks whether the new current symbol is an equals-symbol - if that's the case the declaration is followed by an expression - which means it's an declaration with an assignment otherwise the expression is set to null. 
The method then returns a Declaration, which is a subclass of AstNode.  



% we may also need to document some of the complex parts of the parser, but we will wait with that, until it is actually more or less completed.
%THIS SECTION IS SUPPOSED TO BE A SUBSECTION TO THE PARSER SECTION.
%Should present the interface, which all symbol table implementations must implement.
%Should clarify all considerations through the design process.
\subsection{Symbol Table}
This section concerns with the implementation of the symbol table. 
first, the interface implemented by the the symbol table will be presented.
Second, all considerations through the design process will be discussed.
This will serve as the reasoning behind the final implementation details.
Last, the problematic aspect of method overloading will be touched upon.

\textbf{Symbol Table Interface}\\
\begin{lstlisting}[language=java,label=lis:STInterface,caption=The interface which all symbol table implementations must implement.]
interface ISymbolTable{
	fun openScope()
	fun closeScope()
	fun enterSymbol(name: String, type: TreeNode.Type): Boolean
	fun retrieveSymbol(name: String): Symbol
	fun declaredLocally(name: String): Boolean
}
\end{lstlisting}

\textbf{Considerations}\\
%conclusions.
